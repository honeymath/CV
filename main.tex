
\documentclass[a4paper,10pt]{article}
\usepackage{geometry}
\usepackage{hyperref}
\geometry{margin=1in}

\begin{document}

\title{Curriculum Vitae}
\author{Qirui Li}
\date{}
\maketitle

\section*{Contact Information}
\begin{tabbing}
\hspace{1in} \= \hspace{2in} \kill
\textbf{Cell:} \> (+1)347-453-7631\\
\textbf{Email:} \> \href{mailto:qiruili@postech.ac.kr}{qiruili@postech.ac.kr} \\
%\textbf{Github:} \> \href{https://github.com/honeymath}{honeymath} \\
\textbf{Website:} \> \href{http://qirui.li}{http://qirui.li} \\
\textbf{Address:} \> Math Science Building, 77 Cheongam-Ro, Nam-Gu, Pohang, Gyeongbuk, Korea 37673 \\
\end{tabbing}

\section*{Employment}
\begin{itemize}
    \item \textbf{Pohang University of Science and Technology, Korea} \\
    \textit{Assistant Professor, Mathematics} \hfill 2023 - Present
    \item \textbf{University of Bonn, Germany} \\
    \textit{Postdoctoral Fellow, Mathematics} \hfill 2021 - 2023 \\
    Mentor: Prof. Peter Scholze
    \item \textbf{University of Toronto, Canada} \\
    \textit{Postdoctoral Fellow, Mathematics} \hfill 2018 - 2021 \\
    Mentor: Prof. Stephen Kudla
\end{itemize}

\section*{Education}
\begin{itemize}
    \item \textbf{Columbia University, United States} \\
    \textit{Ph.D., Mathematics} \hfill September 2013 - May 2018 \\
    Advisor: Prof. Wei Zhang
    \item \textbf{Tsinghua University, China} \\
    \textit{B.S., Mathematics} \hfill September 2009 - June 2013 \\
    % GPA: 92.7/100, Rank: 3/78 \\
    % Fellowship in the Tsinghua Xuetang Talents Program \\
    % First Scholarship, \textit{Three consecutive years} \\
    % Thesis Advisor: Ye Tian
\end{itemize}

\section*{Research Interests}
\begin{itemize}
    \item Number Theory and Arithmetic Geometry
    \item Arithmetic Geometry, Gross--Zagier formula
    \item Automorphic Representations, Relative Trace Formulae
    \item Elliptic curves, p-divisible groups, Shimura varieties, Local Shtukas
\end{itemize}
%## Publications
%
%4. **On Keating's results of formal module endomorphisms lifting**  
%   [Arxiv: 1902.10789](https://arxiv.org/abs/1902.10789)
%
%1. **Extensions of vector bundles on the Fargues-Fontaine curve**  
%   Published in: **Journal of the Institute of Mathematics of Jussieu**, Volume 21, Issue 2, March 2022, pp. 487 - 532 https://doi.org/10.1017/S1474748020000183
%   [Arxiv: 1705.00710](https://arxiv.org/abs/1705.00710)
%
%2. **An Intersection formula of CM cycles on Lubin-Tate spaces**  
%   Published in: **Duke Math. J.** 171(9): 1923-2011 (15 June 2022)  https://doi.org/10.1215/00127094-2021-0062
%   [Arxiv: 1803.07553](https://arxiv.org/abs/1803.07553)
%
%3. **A proof of the linear Arithmetic Fundamental Lemma of GL4**  
%   Published in: **Canadian Journal of Mathematics** 74.2 (2022): 381-427  https://doi.org/10.4153/S0008414X20000814 
%   [Arxiv: 1907.00090](https://arxiv.org/abs/1907.00090)
%
%
%5. **Intersections in Lubin-Tate Space and bi-quadratic Fundamental Lemmas**  
%   With Ben Howard, published 2025-06-03: **The American Journal of Mathematics ** https://doi.org/10.1353/ajm.2025.a961344
%   [Arxiv: 2010.07365](https://arxiv.org/abs/2010.07365)
%
%7. **A computational proof for the biquadratic Linear AFL for GL4**  
%   [ArXiv: 2505.22625](https://arxiv.org/abs/2505.22625)
%
%6. **On the Linear AFL: The Non-Basic Case**  
%   With Andreas Mihatsch, 2025-06-23 published in: **Compositio Mathematica** https://doi.org/10.1112/S0010437X24007577;
%   [ArXiv: 2208.10144](https://arxiv.org/abs/2208.10144)
%
%9. **Arithmetic transfer for inner forms of GL2n**  
%   With Andreas Mihatsch, 2025-08-06 published in: **Forum of Mathematics, Sigma**  https://doi.org/10.1017/fms.2025.10034
%   [ArXiv: 2307.11716](https://arxiv.org/abs/2307.11716)

\section*{Selected Publications}
\begin{itemize}
    \item \textbf{Extensions of vector bundles on the Fargues-Fontaine curve} \\
    Published in: Journal of the Institute of Mathematics of Jussieu, Volume 21, Issue 2, March 2022, pp. 487 - 532 \\ \href{https://doi.org/10.1017/S1474748020000183}{doi.org/10.1017/S1474748020000183} \\
    \href{https://arxiv.org/abs/1705.00710}{Arxiv: 1705.00710}
    \item \textbf{An Intersection formula of CM cycles on Lubin-Tate spaces} \\
    Published in: Duke Math. J. 171(9): 1923-2011 (15 June 2022) \\
    \href{https://arxiv.org/abs/1803.07553}{Arxiv: 1803.07553}\\
    \href{https://doi.org/10.1215/00127094-2021-0062}{doi.org/10.1215/00127094-2021-0062}
    \item \textbf{On the Linear AFL: The Non-Basic Case} \\
    With Andreas Mihatsch, 
	Published in: Compositio Mathematica \\
    \href{https://arxiv.org/abs/2208.10144}{ArXiv: 2208.10144}\\
	\href{https://doi.org/10.1112/S0010437X24007577}{doi.org/10.1112/S0010437X24007577}
\end{itemize}
\section*{Other Publications}
\begin{enumerate}
    \item \textbf{A proof of the linear Arithmetic Fundamental Lemma of GL4} \\
    Published in: Canadian Journal of Mathematics 74.2 (2022): 381-427 \\
    \href{https://arxiv.org/abs/1907.00090}{Arxiv: 1907.00090}\\
	\href{https://doi.org/10.4153/S0008414X20000814}{doi.org/10.4153/S0008414X20000814}
    \item \textbf{Intersections in Lubin-Tate Space and bi-quadratic Fundamental Lemmas} \\
    With Ben Howard, published in: The American Journal of Mathematics \\
    \href{https://arxiv.org/abs/2010.07365}{Arxiv: 2010.07365}\\
	\href{https://doi.org/10.1353/ajm.2025.a961344}{doi.org/10.1353/ajm.2025.a961344}
    \item \textbf{Arithmetic transfer for inner forms of GL2n} \\
    With Andreas Mihatsch, published in: Forum of Mathematics, Sigma \\
    \href{https://arxiv.org/abs/2307.11716}{ArXiv: 2307.11716}\\
	\href{https://doi.org/10.1017/fms.2025.10034}{doi.org/10.1017/fms.2025.10034}
    \item \textbf{On Keating's results of formal module endomorphisms lifting} \\
    \href{https://arxiv.org/abs/1902.10789}{Arxiv: 1902.10789}
    \item \textbf{A computational proof for the biquadratic Linear AFL for GL4} \\
	\href{https://arxiv.org/abs/2505.22625}{ArXiv: 2505.22625}
    %\item \textbf{Linear Arithmetic Fundamental Lemma for GL2-- Higher derivatives} \\
    %With Wei Zhang and Andreas Mihatsch, in preparation
\end{enumerate}


%\section*{Publications}
%\begin{enumerate}
%    \item \textbf{Extensions of vector bundles on the Fargues-Fontaine curve} \\
%    Published in: Journal of the Institute of Mathematics of Jussieu, Volume 21, Issue 2, March 2022, pp. 487 - 532 \\
%    \href{https://arxiv.org/abs/1705.00710}{Arxiv: 1705.00710}
%    \item \textbf{An Intersection formula of CM cycles on Lubin-Tate spaces} \\
%    Published in: Duke Math. J. 171(9): 1923-2011 (15 June 2022) \\
%    \href{https://arxiv.org/abs/1803.07553}{Arxiv: 1803.07553}
%    \item \textbf{A proof of the linear Arithmetic Fundamental Lemma of GL4} \\
%    Published in: Canadian Journal of Mathematics 74.2 (2022): 381-427 \\
%    \href{https://arxiv.org/abs/1907.00090}{Arxiv: 1907.00090}
%    \item \textbf{On Keating's results of formal module endomorphisms lifting} \\
%    \href{https://arxiv.org/abs/1902.10789}{Arxiv: 1902.10789}
%    \item \textbf{Intersections in Lubin-Tate Space and bi-quadratic Fundamental Lemmas} \\
%    With Ben Howard, accepted in: The American Journal of Mathematics \\
%    \href{https://arxiv.org/abs/2010.07365}{Arxiv: 2010.07365}
%    \item \textbf{On the Linear AFL: The Non-Basic Case} \\
%    With Andreas Mihatsch, submitted \\
%    \href{https://arxiv.org/abs/2208.10144}{ArXiv: 2208.10144}
%    \item \textbf{A computational proof for the biquadratic Linear AFL for GL4} \\
%    \href{http://qirui.li/draftBFL.pdf}{Draft}
%    \item \textbf{Linear Arithmetic Fundamental Lemma for GL2-- Higher derivatives} \\
%    With Wei Zhang and Andreas Mihatsch, in preparation
%    \item \textbf{Arithmetic transfer for inner forms of GL2n} \\
%    With Andreas Mihatsch, submitted \\
%    \href{https://arxiv.org/abs/2307.11716}{ArXiv: 2307.11716}
%\end{enumerate}
%
\section*{Teaching Experience}
\begin{itemize}
    \item \textbf{Linear Algebra (MATHS2010\_001\_2015\_2)}: Summer 2015, Columbia University
    \item \textbf{Linear Algebra (MATHS2010\_002\_2016\_2)}: Summer 2016, Columbia University
    \item \textbf{College Algebra (MATH1003\_001\_2017\_1)}: Spring 2017, Columbia University
    \item \textbf{Linear Algebra (MATHS2010\_001\_2017\_2)}: Summer 2017, Columbia University
    \item \textbf{Linear Algebra (MAT223)}: Fall 2018, University of Toronto
    \item \textbf{Differential Equations (MAT234)}: Spring 2019, University of Toronto
    \item \textbf{Linear Algebra (MAT224)}: Summer 2019, University of Toronto
    \item \textbf{Calculus (MAT135)}: Fall 2019, University of Toronto
    \item \textbf{Linear Algebra (MAT224)}: Winter 2020, University of Toronto
    \item \textbf{Linear Algebra (MAT224)}: Summer 2020, University of Toronto
    \item \textbf{Calculus (MAT135)}: Fall 2020, University of Toronto
    \item \textbf{Linear Algebra (MAT224)}: Winter 2021, University of Toronto
    \item \textbf{Linear Algebra (MAT224)}: Summer 2021, University of Toronto
    \item \textbf{Linear Algebra (MAT203)}: Fall 2023, POSTECH
    \item \textbf{Elliptic Curves (MAT601)}: Spring 2024, POSTECH
    \item \textbf{Algebraic Number Theory (MAT504)}: Fall 2024, POSTECH
    \item \textbf{Algebra I (MAT501)}: Spring 2025, POSTECH
    \item \textbf{Algebra II (MAT502)}: Fall 2025, POSTECH
\end{itemize}

%\section*{Honors}
%\begin{itemize}
    %\item The 5th New World Mathematics Gold Award for Thesis (September 2019)
    % \item Outstanding Graduates of Beijing
    % \item National Scholarship (2013)
    % \item The 1st Shing-Tung Yau High School Mathematics Award (2008) \\
    % \href{http://www.china-maths.com/shuxue/news/773.htm}{Link}
    % \item 2008 National High School Mathematics Olympiad Provincial First Prize (Provincial Ranking 9) \\
    % \href{https://www.xiaoxiaotong.org/AttachFile/180006/633697704227500000.pdf}{Link}
%\end{itemize}

 %\section*{Computer Science Honors}
% \begin{itemize}
    % \item 2008 National Youth Informatics Olympiad Provincial First Prize (Provincial Ranking 3) \\
    % \href{http://www.xiaoxiaotong.org/AttachFile/180006/633697706512968750.pdf}{Link}
% \end{itemize}

%\section*{Online lecture experiences}
%\subsection*{Linear Algebra}
%	\begin{itemize}
%		\item Example of online lecture video\href{https://honeymath.com/mat/L1/}{https://honeymath.com/mat/L1/}
%		\item Youtube Channel: \href{https://www.youtube.com/channel/UCFgnippxbTlozd2UW8PZmkg}{https://www.youtube.com/channel/UCFgnippxbTlozd2UW8PZmkg}
%	\end{itemize}


%\section*{Projects and Development}
%\subsection*{Honeymath Platform}
%\textit{Description:} An online platform designed to enhance mathematics learning through interactive problem-solving and assignment management. \\
%\textit{Highlights:}
%\begin{itemize}
%    \item Cross-platform compatibility
%    \item Python-based problem creation for interactive, platform-independent problems
%    \item Interactive learning environment with instant feedback
%    \item Comprehensive assignment management with automatic grading
%    \item Role-based access for students, TAs, and instructors
%\end{itemize}
%\textit{Technologies Used:} Python, JavaScript, Vue.js, HTML, CSS, LaTeX, MathJax, Pyodide \\
%\textit{Website:} \href{https://www.honeymath.com}{honeymath.com} \\
%\textit{GitHub:} \href{https://github.com/honeymath/honeyplatform}{github.com/honeymath/honeyplatform}
%
%\subsection*{Linear Algebra Education Tools}
%\textbf{Row Operation Calculator} \\
%Row operation calculator increases learning experience by enabling students to perform matrix operations more quickly. This calculator is capable of finding inverses, eigenvectors, and other purposes. \\
%\textit{Website:} \href{https://honeymath.com/mat/M/}{https://honeymath.com/mat/M/} \\
%\textit{Video Explanation:} \href{https://youtu.be/YK-WYlNLd_8}{https://youtu.be/YK-WYlNLd\_8} \\
%\textit{Technologies used:} JavaScript, Vue.js, HTML, CSS, LaTeX, MathJax\\
%\textbf{Matrix Cross Filler} \\
%Cross-filling decomposition decomposes a matrix into rank 1 matrices. It can be used for LU decomposition and Cholesky decomposition. This project promotes user experience with key-less input design. The initial motivation is to make linear algebra into a game.\\
%\textit{Video Explanation:} \href{https://www.youtube.com/watch?v=wFCsh0A2ZnU}{https://www.youtube.com/watch?v=wFCsh0A2ZnU} \\
%\textit{Website:} \href{https://honeymath.github.io/crossfiller/}{https://honeymath.github.io/crossfiller/} \\
%\textit{Technologies used:} PixiJS
%
%% \subsection*{Game-Related Projects}
%% \textit{JumpChild}. I have made a game when I was a kid. Later on I made this game into python version. I plan to make a web version of this game. \\
%
%%\subsection*{Ongoing Projects}
%%\textbf{PathFlow} \\
%%Pathflow simplifies the data processing process using XPath, JSONPath, and File System Paths. Motivated by a mathematical way of representing symbols, it minimizes the code needed for building a prototype data processing pipeline. \\
%%\textit{Technologies used:} Python, Pyparse
%
%%\subsection*{Textbook of Linear Algebra}
%%Writing a textbook that introduces new methods of teaching linear algebra using projection matrices and Lagrange interpolation polynomials. The book is in progress.
%%
%%% \section*{Social Activities}
%%% \begin{itemize}
%%%     \item Counselor, THU Lijiajie Psychological Consulting Hotline (2010-2012)
%%     \item Volunteer, THU Help Room for undergraduate math study (2010-2012)
%%     \item Website Designer, Columbia University CU Asia (2014-2015)
%%     \item Running Leader, University of Toronto 3M Running Club (2018 - 2020)
%%     \item Counselor, Empower Change Psychological Consulting club in University of Toronto  (2019-2022)
%% \end{itemize}
%
%\section*{Skills}
%\begin{itemize}
%    \item \textbf{Programming Languages:} Python
%    \item \textbf{Web Development:} HTML, CSS, JavaScript
%    \item \textbf{Other:} LaTeX, Markdown
%\end{itemize}
%
\end{document}
%
